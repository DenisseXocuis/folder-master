\documentclass[letterpaper,12pt]{article}
\usepackage[utf8]{inputenc}

\usepackage[T1]{fontenc}
\usepackage{tgtermes} %%%font

\usepackage{geometry}
\usepackage{amsmath}
\usepackage{float}
\usepackage{graphicx}
\usepackage{subcaption}
\usepackage{amssymb}
\usepackage{adjustbox}
\usepackage{wrapfig} %%imagen envuelta por un texto
\usepackage{xcolor}
\usepackage{fancyhdr}
\usepackage{tabularx} %%TABLAS OH YEAH

\title {\textbf{daurencia inf}}
\date{30 de junio del 2023}
\geometry{top=2cm, bottom=2cm, left=2cm, right= 2cm} %%margen
\graphicspath{{images/}}
\parindent=0pt

\begin{document}
\begin{sloppypar} 
\section*{¿Qué es faurecia?}
Faurecia México es una empresa líder en la fabricación de soluciones y tecnologías para la industria automotriz. Inició sus operaciones en el país en 1997 estableciéndose en el Parque Industrial FINSA de la ciudad de Puebla como Sommer Allibert Industries para la fabricación de Interiores.

Desde entonces, Faurecia ha crecido y extendido sus zonas de producción hasta abarcar diferentes estados de la República como Puebla, Querétaro, Guanajuato, San Luis Potosí, Coahuila y Sonora.

Faurecia México se ha convertido en pieza clave para la industria automotriz gracias a su firme compromiso con la digitalización de procesos, en los que cada planta puede dar seguimiento a su producción en tiempo real, a la par de analizar la eficiencia y planes de acción a detalle. 

En febrero de 2022, Faurecia y HELLA presentaron FORVIA, el nombre del recientemente Grupo combinado, tras la finalización exitosa de la adquisición de una participación mayoritaria en HELLA por parte de Faurecia el 31 de enero de 2022.

Fundada en 1997, Faurecia ha crecido hasta convertirse en un jugador importante en la industria automotriz a nivel mundial. Con 266 sitios industriales, 39 centros de I+D y 114,000 empleados en 35 países, Faurecia es un líder global en sus cuatro áreas de negocios: Asientos, Interiores, Clarion Electronics y Movilidad Limpia. La sólida oferta tecnológica del Grupo proporciona a los fabricantes de automóviles soluciones para la Cabina del Futuro y la Movilidad Sustentable. En 2020, el Grupo registró una facturación total de 14,700 millones de euros. Faurecia cotiza en el mercado Euronext Paris y es un componente del índice CAC Next 20.

Actualmente Faurecia tiene su sede en Nanterre, su principal ejecutivo es Patrick Koller. Cuenta con más de 157,000 empleados.

\section*{Historia}
En 1914, Bertrand Faure abrió su primer taller en Levallois-Perret, a las afueras de París, fabricando asientos para tranvías y el metro de París. Desde entonces, Faurecia ha crecido hasta convertirse en uno de los diez principales proveedores mundiales de automoción, trabajando para dar forma a la próxima generación de movilidad en cada momento.

Faurecia, tal y como la conocemos hoy en día, se formó en 1997 con la adquisición de Bertrand Faure por parte de ECIA, propiedad de PSA, para crear una empresa automovilística mundial. En 2021, con la fusión de PSA y FCA y la creación de Stellantis, comenzó un nuevo capítulo en la historia de Faurecia.

Con la adquisición de una participación de control en HELLA, Faurecia y HELLA se han unido para formar FORVIA, un proveedor global de automoción con una cartera de tecnología avanzada y capacidad de innovación. En línea con las megatendencias clave de la industria y las áreas tecnológicas de rápido crecimiento, FORVIA se encuentra en una posición única para aportar soluciones para una movilidad segura, sostenible, avanzada y personalizada.

\subsection*{Línea de tiempo}
\begin{itemize} %%linea del tiempo
    \item 1914: Bertrand Faure abre un taller para fabricar asientos para tranvías y trenes metropolitanos.
    \item 1929: Faure adquiere la patente del muelle Epéda y comienza comienza la producción de sistemas de asientos para el mercado del automóvil.
    \item 1945: Peugeot inicia la producción de componentes de automoción, bicicletas y motocicletas a través de las filiales Aciers et Outillage Peugeot y Peugeot Cycles.
    \item 1987:  Aciers et Outillage Peugeot y Peugeot Cycles se fusionan para formar ECIA; Faure adquiere Delsey y Luchaire.
    
    ECIA había firmado contratos con otros fabricantes de automóviles, como Volkswagen y Renault, que representaron el 18\% y el 11\%, respectivamente, de los ingresos de ECIA ese año, frente al 60\% de PSA Peugeot-Citroën. Otros clientes eran Daimler Benz, Opel Honda y Mitsubishi. Para entonces, las ventas de la empresa habían superado los 1.600 millones de euros, el doble que en 1987. El porcentaje de componentes de automoción en los ingresos totales de ECIA se había más que triplicado durante este tiempo.
    
    \item 1988: Bertrand Faure es adquirido en una LBO respaldada por Michelin, Michel Thierry, Peugeot y otros para bloquear un intento de adquisición por parte de Valeo.
    \item 1991: Faure adquiere Rentrop en Alemania.
    \item 1996:  la complejidad del paquete de adquisición de Faure se volvió en contra de la empresa, cuando Michel Thierry decidió vender su participación del 16,6\% en Faure a ECIA. Este movimiento precipitó las conversaciones entre las dos empresas, que desembocaron en una oferta de adquisición a gran escala de Faure por parte de ECIA a finales de 1997. 
    \item 1998: Faure acuerda ser adquirida por ECIA, formando Faurecia. a fusión, completada en 1998, creó un nuevo gigante francés y europeo, Faurecia. Aunque Faurecia seguía estando controlada por PSA Peugeot-Citroën, que poseía más del 70\% de sus acciones y seguía siendo su principal cliente, la nueva empresa se constituyó como una compañía de funcionamiento independiente.
    \item 1999: Faurecia adquiere AP Automotive Systems en Estados Unidos.
    \item 2001: Faurecia adquirió la división de automoción de Sommer Allibert, formando así el principal proveedor europeo de componentes de automoción.
    \item 2003: Faurecia consigue un contrato de 2.000 millones de dólares para la producción de componentes de cabina para Chrysler en Estados Unidos.
    \item 2004: Faurecia avanzó en su esfuerzo por diversificar su base de clientes. En 2005, la empresa había firmado contratos con un número cada vez mayor de fabricantes de automóviles, entre ellos Volvo, DaimlerChrysler, Saab y BMW. A mediados de la década, Faurecia también se convirtió en uno de los principales proveedores de General Motors. En octubre de 2004, la empresa inició la producción de cabinas para el nuevo Pontiac G6, imponiéndose a los favoritos Johnson Controls y Lear por el contrato. Más tarde, en noviembre de 2004, Faurecia recibió un contrato de casi 2.000 millones de dólares para fabricar paneles de instrumentos, paneles de puertas, consolas centrales y otros componentes para las cabinas de los automóviles estadounidenses de Chrysler. Faurecia se había consolidado como una fuerza clara y creciente en el mercado mundial de componentes de automoción.
    \item 2005: En su lugar, Faurecia empezó a ampliar su presencia internacional, abriendo nuevas plantas en Brasil y Polonia en 1998. Al año siguiente, la empresa se centró en Norteamérica y adquirió AP Automotive Systems (APAS), el tercer productor de sistemas de escape del mercado. La presencia de Faurecia en Norteamérica también se vio impulsada ese año, cuando se le adjudicó un importante contrato de producción para General Motors. La empresa no tardó en ampliar su capacidad de producción en Norteamérica, abriendo o adquiriendo fábricas hasta alcanzar casi 30 instalaciones en ese mercado en 2005.
    \item 2021:con la fusión de PSA y FCA y la creación de Stellantis, comenzó un nuevo capítulo en la historia de Faurecia. Groupe PSA, la multinacional francesa que producía las marcas Peugeot, Citroën, DS, Opel y Vauxhall, era accionista mayoritario de Faurecia. Cuando Groupe PSA se fusionó con Fiat Chrysler para formar Stellantis, Faurecia se independizó y adquirió una participación de control en HELLA
\end{itemize}

\section*{¿Qué hace faurecia?}
Faurecia está especializada en la producción de piezas para automóviles: asientos, paneles de instrumentos, paneles de puertas, tecnologías para reducir las emisiones de C02. La empresa también está apostando fuerte por el hidrógeno. Tiene previsto construir una planta de producción de tanques de hidrógeno.

Faurecia SA es un holding dedicado a la fabricación y suministro de componentes de automoción. Opera a través de los siguientes segmentos de negocio: Faurecia Automotive Seating, Faurecia Emissions Control Technologies, Faurecia Interior Systems y Faurecia Automotive Exteriors. Faurecia Automotive Seating se dedica al diseño de asientos para vehículos, la fabricación de armazones de asientos y mecanismos de ajuste, y el montaje de unidades de asiento completas. El segmento Faurecia Emissions Control Technologies se dedica a la fabricación y diseño de sistemas de escape. El segmento Faurecia Interior Systems fabrica y diseña paneles de instrumentos, paneles y módulos de puertas y componentes acústicos. El segmento Faurecia Automotive Exteriors se dedica al diseño y fabricación de frontales y módulos de seguridad. Faurecia fue fundada el 1 de julio de 1929 y tiene su sede en Nanterre, Francia. 

https://www.forbes.com/companies/faurecia/?sh=59a126075f07




\section*{Acerca de FORVIA}
FORVIA comprende la tecnología complementaria y las fortalezas industriales de Faurecia y HELLA. Con más de 300 sitios industriales y 77 centros de I+D, 150,000 personas, incluidos más de 35,000 ingenieros en más de 40 países, FORVIA ofrece un enfoque único e integral para los desafíos automotrices de hoy y de mañana. Compuesto por 6 grupos de negocio con 24 líneas de productos y una sólida cartera de propiedad intelectual de más de 14,000 patentes, Forvia se enfoca en convertirse en el socio de innovación e integración preferido para los OEM en todo el mundo. FORVIA pretende ser un agente de cambio comprometido con prever y hacer realidad la transformación de la movilidad.



\section*{Hella}
¿Qué le parece Hella?
Esta empresa alemana es líder mundial en tecnología de iluminación para automóviles. También está especializada en componentes electrónicos. En 2021, el proveedor automovilístico generó unas ventas de 6.500 millones de euros y cuenta con algo más de 36.000 empleados en todo el mundo.

Fundada en Alemania en 1899, HELLA es uno de los principales proveedores de tecnología punta en iluminación y electrónica para vehículos. Al mismo tiempo, la empresa cubre una amplia cartera de servicios y productos para el negocio de recambios y talleres, así como para fabricantes de diversos vehículos especiales, por ejemplo de los sectores de la construcción, la agricultura o el transporte. HELLA es una empresa que cotiza en bolsa y que estuvo participada mayoritariamente por las familias Hueck y Roepke antes de que Faurecia se convirtiera en accionista mayoritario.

\section*{Adquisición de hella }
¿Qué significa esta adquisición?
Faurecia ha adquirido en total alrededor del 79,5\% de las acciones de Hella. Con la adquisición de su homólogo alemán, Faurecia cambia también su nombre por el de Forvia. Con unas ventas combinadas de más de 20.000 millones de euros y 155.000 empleados, Forvia se convierte en el séptimo proveedor mundial de la industria automovilística.

Los dos grupos seguirán funcionando como dos entidades jurídicas independientes, pero operarán en determinados sectores. No obstante, se comunicarán bajo el nombre de Forvia, utilizado como "nombre paraguas".



\section*{Ambiciones de forvia}
¿Cuáles son las ambiciones de Forvia?
El grupo piensa a lo grande. "Uno de cada dos coches fabricados en el mundo estará equipado con productos Forvia", prevé Patrick Koller, Director General de Faurecia. Con la compra de Hella, la empresa francesa adquiere tecnologías, sobre todo en iluminación, para seguir desarrollándose en el mercado del coche eléctrico. "En 2025, el volumen de negocios de la electrónica ascenderá a 7.000 millones de euros", afirma Patrick Koller.

La empresa también apuesta fuerte por su "cabina del futuro", un salpicadero inmersivo y conectado para vehículos. "Como Forvia, estamos dando forma a una movilidad segura, sostenible, tecnológica e individualizada, hoy y para las generaciones futuras", continúa. El nuevo gigante automovilístico aspira a facturar 33.000 millones de euros en 2025.

Al unir Faurecia y HELLA viene la creación de FORVIA, se ha creado un nuevo líder global en tecnología de automoción con la experiencia y la pasión para el futuro de la movilidad, en línea con las tendencias que están transformando la industria del automóvil : electrificación, conducción automatizada y autónoma, y experiencias personalizadas a bordo.

La visión y la estrategia de FORVIA encarnan un grupo comprometido a impulsar el cambio en la movilidad la tecnología complementaria y las fortalezas industriales de Faurecia y HELLA. Combinadas, estas dos compañías tienen una amplia cartera de soluciones de automoción con posiciones de liderazgo en tecnologías clave y segmentos de rápido crecimiento, creando las condiciones ideales para un crecimiento sostenible y rentable.

https://www.forvia.com/who-we-are/our-companies

\section*{Faurecia accionista de stellantis }


Stellantis, Michelin y Faurecia, socios en el negocio de las pilas de combustible

\section*{Faurecia vende parte de su negocio de vehículos comerciales en Europa y EEUU a Cummins por 142 millones}


\section*{Bibiografía}
https://www.faurecia-mexico.mx/acerca-de-nosotros/descubre-faurecia-mexico

\end{sloppypar}
\end{document}

