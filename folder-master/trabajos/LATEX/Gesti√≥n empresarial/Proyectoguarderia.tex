\documentclass[letterpaper,12pt]{article}
\usepackage[utf8]{inputenc}

\usepackage[T1]{fontenc}
\usepackage{tgtermes} %%%font

\usepackage{geometry}
\usepackage{amsmath}
\usepackage{float}
\usepackage{graphicx}
\usepackage{subcaption}
\usepackage{amssymb}
\usepackage{adjustbox}
\usepackage{wrapfig} %%imagen envuelta por un texto
\usepackage{xcolor}
\usepackage{fancyhdr}
\usepackage{tabularx} %%TABLAS OH YEAH

\title {\textbf{Proyecto}}
\author{Lara Xocuis Martha Denisse}
\date{21 de septiembre de 2023}
\geometry{top=2cm, bottom=2cm, left=2cm, right= 2cm} %%margen
\graphicspath{{images/}}
\parindent=0pt

\begin{document}
\maketitle

%%%%%%%%%%%%%%%%%%%%%%%%%%%%%%%%%%%%%%%%%%%%%%%%%%%%%%%%%%%%%%%%%%%%%%%%%%%%
\begin{sloppypar}
\section{¿Qué es una PyMe?}
Este término se refiere a las pequeñas y medianas empresas que tienen un número reducido de empleados y un volumen de facturación moderado. Las pymes se clasifican en tres tipos y tienen características específicas que te ayudarán a identificarlas. Las pymes son las iniciativas por la cual los emprendedores incursionan en el mundo empresarial, ya que, la inversión inicial para abordar una empresa de pequeña escala es admisible para muchos de ellos.
\vspace{0.3cm}\\ 
Una pyme (pequeña y mediana empresa) es una entidad organizativa que se identifica por poseer un número relativamente bajo de trabajadores, unas actividades comerciales limitadas, y un ingreso económico no superior a las grandes empresas.

Las pymes son un tipo de empresa que se clasifican de acuerdo a su escala, que incluyen la cantidad de empleados, su expansión empresarial y su poder económico. Estas 3 variables definen en parte si una empresa pertenece a una categoría u otra. En este caso, una empresa es una pyme si:

\begin{itemize}
    \item Posee menos de 250 trabajadores 
    \item Tiene una facturación anual menor a 50 millones de euros 
    \item Su expansión empresarial es menor a la de una gran empresa
\end{itemize}
\subsection{Características de una PyMe}
\begin{itemize}
    \item Las pymes constituyen la mayoría del mercado del comercio común.
    \item Su presencia en las empresas industriales es muy baja, debido a la gran cantidad de dinero que necesitan para entrar.
    \item Tienen un valor esencial en la economía de un país.
    \item No suelen tener presencia internacional. Aunque con la digitalización masiva, muchas empresas tecnológicas prestan sus servicios internacionalmente.
    \item Son muy diversas de acuerdo a su actividad económica.
\end{itemize}
\subsection{Ventajas de una PyMes}
\begin{itemize}
    \item Atacan a nichos más específicos 
    \item Tienen más flexibilidad en redirigir y modificar su modelo de negocio.
    \item Las actividades comerciales de la empresa pueden ser atendidas por un número bajo de empleados.
    \item Dependiendo del negocio, el capital necesario para levantar una empresa pequeña puede ser cubierta por un número bajo de emprendedores.
    \item Las pymes son la principal fuente de empleos de cualquier nación.
    \item Los fundadores de la empresa tienen una relación más cercana con los clientes, con lo cual, pueden desarrollar y describir mejor a su buyer persona.
\end{itemize}
\subsection{Desventajas de las PyMes}
\begin{itemize}
    \item Las pymes cuentan con pocos recursos monetarios y baja cantidad de trabajadores. Por esta razón, es común observar empleados realizar actividades que no están relacionadas a sus habilidades por las cuales fueron contratados.
    \item Si bien las pymes pueden abordar nichos pocos competitivos, lo cierto es que la mayoría de los mercados están saturados, y el porcentaje de éxito de la empresa es relativamente bajo.
    \item Los márgenes de ganancias de una empresa pequeña y mediana pueden ser ajustados en la mayoría de los casos.
    \item Su probabilidad de expansión es limitada, aunque no imposible.
    \item El salario que pueden ofrecer es bajo en comparación a las grandes corporaciones.
\end{itemize}
\subsection{Importancia de las PyMes}
Las pymes representan un alto porcentaje de la totalidad de empresas que existen en el mundo. Con lo cual, las actividades económicas y la cantidad de empleos disponibles provienen mayormente de las empresas medianas y pequeñas.

Esto significa que, gran parte de la economía de un país está constituida por la intervención de estas empresas, haciendo que su presencia sea un pilar esencial para el sustento financiero de una nación.

Promueve el emprendimiento, haciendo que nuevas empresas se creen y se establezcan en el mercado, ofreciendo nuevas fuentes de empleos y generando más variedad de organizaciones en los distintos ámbitos de comercio.

\subsection{Tipos de PyMes}
El tipo de las pymes depende del número de trabajadores e ingresos anuales, y se clasifican en microempresas, empresas pequeñas y empresas medianas.

\section{¿Qué es una MiPyMe?}
Las MiPyMes incluyen tanto a las micro, pequeñas y medianas empresas, y son de suma importancia para nuestro país dada su aportación a la economía, mediante la generación de empleos, ingresos y abastecimiento.

A diferencia del término pymes, al hablar de mipymes se incluyen tanto a las micro, pequeñas y medianas empresas. Veamos la clasificación de estas empresas a continuación:
\vspace{0.3cm}\\ 
\textbf{Microempresas: }Usualmente, están conformadas por no más de 10 personas y tienden a ser empresas familiares. Son creadas por personas que, en un inicio, no cuentan con un capital muy grande y su monto máximo de venta es de 4 millones de pesos anuales.
\vspace{0.3cm}\\
\textbf{Pequeña empresa:} A diferencia de las microempresas, las pequeñas empresas pueden sumar hasta un total de 50 personas a su equipo de trabajo. Y, gracias a su capital, tienen la oportunidad de crecer un poco más e incluso expandirse a través de sucursales. Dependiendo del sector en el cual operen pueden llegar a facturar hasta \$100 millones MXN al año.
\vspace{0.3cm}\\
\textbf{Mediana empresa}: Este tipo de empresas son más grandes que las dos anteriores, ya que pueden contar con 51 hasta 100 trabajadores, y tienen un nivel de organización más definido. Otro diferenciador es el tipo de obligaciones que deben cumplir y los derechos laborales que les deben otorgar a sus colaboradores. Las empresas medianas suelen abastecer a un mercado más amplio, que puede ser local, regional e, incluso, nacional o internacional. 

\section{Indicadores}
Para optimizar los procesos dentro de una PyMe, es fundamental que se puedan medir los resultados que se obtienen de estos y se comparen con los objetivos, es necesario conocer los diferentes tipos de indicadores que ayudan a lograrlo.

\begin{enumerate}
  \item Indicadores comerciales y de satisfacción al cliente: busca triunfar en el mundo de los negocios, estar consciente que su empresa debe resolver una necesidad de consumidores. 
  Sobre este tema existen diferentes tipos de indicadores, por ejemplo: 
  \begin{enumerate}
  	\item Coste de adquisión de clientes (CAC)
  	\item Retención de compradores 
  	\end{enumerate}
\item Indicadores financieros: Si la empresa no produce suficientes ingresos para mantenerse estable y crecer de forma organizada.
	\begin{enumerate}
		\item Facturación 
		\item Margen de ganancia 
		\item Flujo de caja
	\end{enumerate}
\item Indicadores de recursos humanos 
	\begin{enumerate}
		\item Ausencia laboral
		\item Rotación
		\item Productividad del empleado
	\end{enumerate}
\item Indicadores estratégicos: basados en los objetivos generales y específicos de tu organización, puedan medir el nivel de resultados globales de tu negocio en un periodo de tiempo específico, es fundamental.
\end{enumerate}

Según las metas que tengas para tu negocio, podrían ser a nivel de facturación, posicionamiento en el mercado, obtención de usuarios o el valor de tu PYME en general.

\section{Organización}
num de empleados, que hace, que servicios hace, horario, dias. 
Contexto. 




\subsection{Bibliografía}
Chavez, J. (2022, March 2). Ceupe. Ceupe. https://www.ceupe.com/blog/pyme.html
\vspace{0.3cm}\\ 
BBVA MEXICO \& BBVA. (2023, January 10). ¿Qué son las Pymes y qué tipos hay? BBVA México. https://www.bbva.mx/educacion-financiera/creditos/que-es-una-pyme.html

\end{sloppypar}
\end{document}