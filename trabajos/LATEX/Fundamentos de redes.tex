\documentclass[letterpaper,12pt]{article}
\usepackage[utf8]{inputenc}

\usepackage[T1]{fontenc}
\usepackage{tgtermes} %%%font

\usepackage{geometry}
\usepackage{amsmath}
\usepackage{float}
\usepackage{graphicx}
\usepackage{subcaption}
\usepackage{amssymb}
\usepackage{adjustbox}
\usepackage{wrapfig} %%imagen envuelta por un texto
\usepackage{xcolor}
\usepackage{fancyhdr}
\usepackage{tabularx} %%TABLAS OH YEAH

\title {\textbf{Fundamentos de redes de computadoras}}
\author{Lara Xocuis Martha Denisse}
\date{31 de agosto de 2023}
\geometry{top=2cm, bottom=2cm, left=2cm, right= 2cm} %%margen
\graphicspath{{images/}}
\parindent=0pt

\begin{document}
\maketitle
\newpage
%%%%%%%%%%%%%%%%%%%%%%%%%%%%%%%%%%%%%%%%%%%%%%%%%%%%%%%%%%%%%%%%%%%%%%%%%%%%
\begin{sloppypar}
\section{Red}
Conjunto de computadoras y/o dispositivos moviles con capacidad de interconexión.

\section{Comunicación de datos}
Transmisión de información codificada de un punto a otro por medio de sistemas de transmisión

\section{¿Cómo las redes nos afectan a nuestra vida diaria?}

\section{Arquitectura del internet}
Abarca con los dispositivos que están para dar el servicio
\begin{itemize}
    \item ISP - proveedor de servicios de internet, megacables, telmex, etc 
    \item DSL - Línea de suscripción digital 
    \item DSLAM - Multiplexor de Acceso a la línea de suscriptor digital, usuarios que están conectados en un servicio en una cierta área
    \item CMTS - Sistema de terminación del módem de cable 
    \item IXP - Punto de intercambio en internet 
    \item FTTH - Fibra para el hogar 
    \item Modem - dispositivo que realiza conversiones entre bits digitales y señales analógicas 
    \item POP - Punto de presencia 
    \item Host (anfintrión) - Computadoras u otros dispositivos (tabletas, smartphone, y portátiles ) conectados a una red que proveen y utilizan servicios de ella
\end{itemize}
\section{La función, los componentes y los desafíos del networking de datos}
Elementos que conforman una red
\vspace{0.3cm}\\ 
\textbf{Dispositivos}: Se utilizan para efectuar la comunicación entre sus elementos 
\vspace{0.3cm}\\ 
\textbf{Medio}
La manera en que los dispositivos se conectan entre si 
\vspace{0.3cm}\\
\textbf{Mensajes} 

\section{Red Convergente}
Tipo de red que puede transmitir voz, video y datos a través de la misma red

\section{Conceptos preliminares}
\subsection{Sistema distribuido}
Colección interconectada de computadoras transparentes al usuario.  

\section{Segmentación}
Los datos transporta en pequeños "bloques" (segmentos). Cada segmento se etiqueta con el número de aplicación.
\begin{itemize}
    \item Web = 80 
    \item FTP = 20/21 
    \item Telnet = 23
    \item Correo SMTP = 25 
    \item POP = 110 
\end{itemize}

\subsection{Características relacionadas con el diseño de Arquitecura de Red }
\begin{itemize}
    \item Tolerancia a fallas 
    \item Escalabilidad 
    \item Calidad de servicio 
    \item Seguridad
\end{itemize}

\textbf{Hardware: routers, switches}

\textbf{Software : IOS(implementación de protocolor)}
\begin{itemize}
    \item Aplicación: DNS, HTTP, correo, etc 
    \item Enrutamiento, Ipv4, IPv6 
    \item Enlace datos: Ethernet, protocolos wireless
\end{itemize}

\section{Medios de red}
Canal por el cual se transmite el mensaje, comunicación puede atravesar diferentes medios.

\section{Coberturas de redes}
\subsection{Clasificación}
\begin{itemize}
    \item PAN (Red de área personal)
    \item MAN (Redes de área metropolitana)
    \item LAN (Redes de área local)
    \item WLAN (Redes inalámbricas de área local)
\end{itemize}

\section{Tipos de comunicación NFC}

\subsection{Pasiva}
Uno de los dispositivos crea un campo y el otro se aprovecha de él para transferir datos
\section{Usos cotidianos del NFC}
\begin{itemize}
    \item Pago mediante uso de los smartphone
    \item Conexión entre usuarios y a juegos mediante dispositivos móviles de juegos electrónicos 
    \item Envío de facturas electrónicas y encuestas de satisfacción facilitando la comunicación con el usuario 
    \item Lectura de tarjetas cliente en aplicaciones de fidelización
    \item Descuentos y promociónes mediante carteles y posters comerciales adaptados a la tecnología NFC 
\end{itemize}
La tasa de \textbf{transferencia} que puede alcanzar el NFC seríaentre los 106Kb / segundos, 212 Kb / segundos, 424 Kb / segundos, 848 Kb / segundos.

Por lo que su enfoque más que para la transmisión de grandes cantidades de datos es para cominicación instantánea e identificación y validación de equipos/personas.

\section{¿Qué es el airdrop?}
Es un servicio Ad-Hoc que Apple, funciona tal cual una carpeta virtual conectada a otros dispositivos de Apple. 

Airdrop hace uso del Bluetooth y del Wifi para detectar dispositivos y transferir los archivos, por lo tanto es necesario tener ambas conexiones activas.

El Bluetooth se utiliza para detectar dispositivos y establecer la conexión, mientras que la transferencia de archivos se realiza mediante la conexión Wifi mucho más rápida y con mayor ancho de banda.

\section{Nearby}
Es la versión de Airdrop de Apple pero para Android, Nearby Sharing es un modo muy sencillo de enviar archivos de un smartphone a otro sin preocuparse por qué tecnología usar con la condición de que ambos dispositivos móviles se encuentren en las proximidades. 

Usa el estándar Bluetooth Low Energy para encontrar los smartphone cercanos y mandar la petición de que se puede enviar el archivo. Si el usuario en el otro smartphone acepta recibirlo, entonces el sistema elige cual es la mejor tecnología para enviar el archivo: Bluetooth o Wi-Fi.

\section{Red de Área local - LAN}

\section{WLAN}
Es lo mismo que LAN pero del mismo modo se puede establecer una \textbf{conexión inalámbrica}

\section{MAN - Red de área metropolitana}
Abarca un área más grande. 

\section{Cobertura WAN}
Técnicamente es el internet, redes de área amplia.
\newpage
\section*{UNIDAD 2}
\section{MODELO DE INTERCONEXIÓN DE SISTEMAS ABIERTOS (OSI)}
2.1 Protocolos de comunicación de red

2.1.1 Estructura de protocolo de internet (ip)

\subsection{Protocolo}
Reglas establecidad para el trato social o es el conjunto de reglas de formalidad establecidas para actos diplomáticos y ceremonias oficiales, tener la base de un conjunto de pasos a seguir para hacer algo o para respetar algún tipo de reglamento

\subsection{Definición técnica de protocolo}
Un \textbf{protocolo} es un conjunto de reglas usadas por computadoras para comunicarse unas con otras a través de una red

\subsection{Protocolo de red de comunicación}
Es un conjunto de reglas que gobierna el intercambio ordenado de datos dentro de la red.

\subsection{Estructura de protocolo de internet (IP)}
IP significa Protocolo de Internet, su función es identificar de manera lógica y jerárquica a una interfaz en red de un dispositivo que utilice el protocolo IP o, que corresponde al nivel de red del modelo TCP/IP o modelo OSI.

Va a depender del sistema operativo y los drivers para poder conectarse. 

\subsection{Clases de IP que existen}
\textbf{CLASE A:} de 10.0.0.0 a 10.255.255.255, que son utilizadas generalmente para grandes redes privadas, por ejemplo, de alguna empresa trasnacional. Es para redes muy grandes
\vspace{0.3cm}\\ 
\textbf{CLASE B:} de 172.16.0.0 a 172.31.255.255, que son usadas para redes medianas, como de alguna empresa local, escuela o universidad. Redes locales
\vspace{0.3cm}\\
\textbf{CLASE C: }de 192.168.0.0 hasta 192.168.255.255, que son para redes más pequeñas o redes domésticas.

\subsection{IP pública}
Estas son indispensables para conectarse a internet, y son visibles para cualquier internauta, y suele ser la que tiene tu router o tu módem 

\subsection{IP dinámica}
Es la dirección IP que va a cambiar cada vez que el dispositivo establece una conexión a internet o cuando se llega a apagar el modem o router en donde se encuentra conectado.

\subsection{IP estática}
Es la dirección IP asignada a un dispositivo de por vida, es decir, jamás cambiará

\subsection{VoIP}
Herramienta que ciertos modems tienen donde el teléfono con conexión a internet, este teléfono se conecta al router o al modem y permitirá el teléfono inalámbrico con internet.

\subsection{IPTV}
Servicio de entretenimiento de canales de paga pero transmitidos desde el internet.

\subsection{Instituto de Ingeniería Eléctrica y Electrónica}
En inglés se conoce como The Institute of Electrical and Electronics Enginners. Es una asociación técnico-profesional mundial dedicada a la estandarización y al desarrollo en áreas técnicas.

\subsubsection{Estándar IEE 802}
Protocolos o estándares de comunicación 
\begin{itemize}
    \item Estándar IEE 802.3
    \item Estandar IEE 802.11 (A,B,G,N - AC Y AH)
\end{itemize}

\subsubsection{Estándar IEE 802.3}
Es mejor conocido como Ethernet, es hasta ahora el tipo más común de la red LAN alámbrica

\subsubsection{Estándar IEE 802.11}
Se utiliza más para las redes LAN inalámbricas, es mejor conocido como WiFi, la cual dependiendo de su velocidad será su protocolo

\section{¿Qué es la banda ancha?}
La conexión de internet de alta velocidad conocida como Banda Ancha (amplio ancho de banda) se define con velocidades de descarga de al menos 768 Kbps y velocidades de carga al menos 200 Kbps. 

La diferencia entre las velocidades de descarga y las de carga puede explicarse como sigue: la velocidad de descarga se refiere a la tasa en la que se transfieren los datos digitales desde el internet a tu computadora, mientras que la velocidad de carga es la tasa en la que se transmiten los datos en línea desde tu computadora a internet.

\section{Banda en giga hertz GHz}
La letra G de la wifi se refiere a las bandas de frecuencia de radio.

El 2.4GHz significa 2.4 GigaHerz, mientras que 5Ghz significa 5GHz.
\section*{Estándar IEE 802}
\section{Estándar IEEE 802.11 A}
\begin{itemize}
    \item Conocido como Wi-Fi5
    \item Funciona con conexiones de hasta 54 Mbps 
    \item Opera en la banda de 5 GHz
\end{itemize}
\section{Estándar IEE 802.11 B}
\begin{itemize}
    \item Funciona con conexiones de hasta 11 Mbps
    \item Opera en la banda de 2.4GHz
\end{itemize}
\section{Estándar IEE 802.11 G}
\begin{itemize}
    \item Funciona con conexiones de hasta 54 Mbps 
    \item Opera en la banda de 2.4GHz
\end{itemize}
\section{Estándar IEE 802.11 N}
\begin{itemize}
    \item Funciona con conexiones de hasta 600 Mbps 
    \item Opera en la banda de 2.4GHz y 5 GHz
\end{itemize}
\section{Estándar 802.11 B/G/N}
\begin{itemize}
    \item Funciona con conexiones de hasta 54Mbps 
    \item Opera en la banda de 2.4 GHz
\end{itemize}
\section{Estándar IEEE 802.11 AC}
\begin{itemize}
    \item Funciona con conexiones de hasta 1300 Mbps 
    \item Opera en la banda de 5GHz
\end{itemize}
\section{Estándar IEEE 802.11 AH}
Publicado en 2017 
\begin{itemize}
    \item También conocido como Wi-Li Halow 
    \item Estándar orientado al "Internet de las cosas" (Internet of things) abrebiado IOT y en español seria IDC. Por el cual su beneficio será la cobertura y no la velocidad.
    \item Funciona con conexiones de hasta 150 Kbps
\end{itemize}
\end{sloppypar}
\end{document}