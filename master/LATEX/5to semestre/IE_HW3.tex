\documentclass[letterpaper,12pt]{article}
\usepackage[utf8]{inputenc}
\usepackage{charter}
\usepackage{geometry}
\usepackage{amsmath}
\usepackage{float}
\usepackage{graphicx}
\usepackage{subcaption}
\usepackage{amssymb}
\usepackage{adjustbox}
\usepackage{wrapfig} 
\usepackage{xcolor}
\usepackage{fancyhdr}
\usepackage{tabularx}
\usepackage{fancyhdr}
\usepackage{comment}

\geometry{top=2cm, bottom=2cm, left=2cm, right= 2cm} %%margen
\graphicspath{{images/}}
\parindent=0pt

\begin{document}
%%para contador y eso
\thispagestyle{empty}
\newpage
\setcounter{page}{1}
\pagestyle{headings}
%%%%%%%%%%%%%%%%%%%%%%%%%%%%%%%%%%%%%%%%%%%%%%%%%%%%%%%%%%%%%%%%%%%%%%%%%%%%
\begin{sloppypar} 
    %%%portadita
    \begin{titlepage}
        \fancyhead{R}{
        \includegraphics[width=0.1\linewidth]{logoUV.png}
        }
        \hspace{2.5cm}
        {\bfseries\LARGE Universidad Veracruzana \par}
        \hspace{2cm}
        {\scshape\Large Facultad de Ingeniería Eléctrica y Electrónica \par}
        \begin{center}
            \vspace{7cm}
            {\itshape\huge INGENIERÍA ECONÓMICA \par}
            {\large Lara Xocuis Martha Denisse\par}
            {\large S22002213 \par}
            \vfill
            {\Large September 16, 2024 \par}
        \end{center}
    \end{titlepage} 

%%%%%%%%%%%%%%%%%%%%%%main template%%%%%%%%%%%%%%%%%%%%%%
\section*{EJERCICIOS DE INGENIERÍA ECONÓMICA ENTREGA 03}
\subsection*{INSTRUCCIONES: Responde a las preguntas del siguiente CASO de estudio}
\subsection{Análisis de Costos en el Desarrollo de un Dron Multifuncional}

%%%%%%%%%%%%%%%%%%info y así%%%%%%%%%%%%%%%%%%%
\textbf{\textit{CONTEXTO:}}
\begin{figure}[H]
    \centering 
    \includegraphics[width=0.7\textwidth]{dronetech.png}
\end{figure}
DroneTech es una startup fundada por tres recién graduados: Elena (Ing. Electrónica), Marco (Ing. Mecatrónica) y Sofía (Ing. Informática). La empresa se especializa en el desarrollo de drones para aplicaciones industriales y de investigación. Después de un año de I+D, han creado un prototipo de dron
multifuncional llamado "VersiDron", diseñado para adaptarse a diversas aplicaciones cambiando sus módulos.
\vspace{0.3cm}\\ 
\textbf{\textit{OBJETIVO DEL CASO:}}
\textit{Aplicar diversos conceptos de costo en un escenario realista que combina aspectos de ingeniería electrónica, mecatrónica e informática en el contexto del desarrollo y producción de drones.}
\vspace{0.3cm}\\ 
\textbf{\textit{ACTIVIDADES A REALIZAR:}}
\vspace{0.3cm}\\ 
\textit{Clasificación de costos:}
\begin{itemize}
    \item Identificar y clasificar los diferentes tipos de costos (fijos, variables, directos, indirectos, hundidos, etc.)
    \begin{itemize}
        \item \textbf{Fijos:} Gastos generales y administrativos y costo de instalaciones de producción
        \item \textbf{Variables:} Componentes de estructura base, costos de módulos intercambiables y mano de obra para ensamblaje.
        \item \textbf{Directos:} Costo de modulos, mano de obra por ensamblaje
        \item \textbf{Indirectos}: Marketing y ventas
        \item \textbf{Hundidos}: Desarollo de software
    \end{itemize}
    \newpage
    \item Explicar cómo cada tipo de costo afecta la toma de decisiones en la producción de VersiDron.\\ Los costos fijos e indirectos son fundamentales para comprender la estructura general de costos y la rentabilidad global, los costos variables y directos afectan directamente a las decisiones sobre volúmenes de producción, selección de unidades y fijación de precios. Para fijar precios competitivos y cubrir todos los costos relevantes, hay que tener en cuenta los costos de marketing y ventas.
\end{itemize}
\textit{Análisis de Costo-Volumen-Utilidad:}
\begin{itemize}
    \item Calcular el punto de equilibrio para cada configuración de VersiDron (con
    diferentes módulos).
    \begin{center}
    Suma de costos fijos:
    $\$100,000 + \$50,000 = \$150,000$ \\ 
    Suma de costos variables por cada modulo: $\$2,000 + (modulo) + \$500$ \\ 
    Y suponiendo que el precio de venta es de 5mil pesos:
    \begin{itemize}
        \item Camara: \$2,000 + \$800 + \$500 = \$3,300 \\ Punto de equilibrio:
        
        $\displaystyle \frac{\$150,000}{\$5,000 - \$3,300} = 88$ unidades
        \item Sensor LiDAR:  \$2,000 + \$1,500 + \$500 = \$4,000 \\ Punto de equilibrio:
        
        $\displaystyle \frac{\$150,000}{\$5,000 - \$4,300} = 150$ unidades
        \item Brazo robótico:  \$2,000 + \$1,200 + \$500 = \$3,700 \\ Punto de equilibrio:
        
        $\displaystyle \frac{\$150,000}{\$5,000 - \$3,700} = 115$ unidades
        \item Sensor multiespectral:  \$2,000 + \$1,000 + \$500 = \$3,500 \\ Punto de equilibrio:
        
        $\displaystyle \frac{\$150,000}{\$5,000 - \$3,500} = 100$ unidades
    \end{itemize}
    \end{center} 
    \item Determinar cómo cambia el punto de equilibrio si se consideran los tres pedidos juntos. 
    \begin{center}
    Pedidos totales: 50 + 10 + 100 = 160 unidades
    \begin{itemize}
        \item Camara: \$3,300 * 160 = \$528,000
        \item Sensor LiDAR \$4,000 * 160 = \$640,000
        \item Brazo robótico \$3,700 * 160 = \$592,000
        \item Sensor multiespectral \$3,500 * 160 = \$560,000
    \end{itemize}
    \end{center}
\end{itemize}
\newpage
\textit{Costeo Basado en Actividades (ABC):}
\begin{itemize}
    \item Desarrollar un sistema de costeo ABC para la producción de VersiDron. \\  Diseñar, adquirir materiales, montar, probar, empacar, marketing, y administración, factores que causan los costos, como horas de desarrollo, órdenes de compra, unidades ensambladas, etc.
    \item Comparar los resultados del costeo ABC con un enfoque de costeo tradicional.
    \begin{center}
        \begin{tabular}[H]{|c|c|} \hline 
            ABC & Tradicional \\ \hline
            Visión precisa de los costos & Asigna costos generales\\ 
                & y fijos de manera uniforme\\ \hline
            Mejor comprensión & Menos exacto\\ \hline
        \end{tabular}
    \end{center}
\end{itemize}
\textit{Análisis de Costos Marginales:}
\begin{itemize}
    \item Calcular el costo marginal de producir unidades adicionales para cada
    cliente.
    \begin{center}

        \begin{itemize}
            \item Camara: \$3,300
            \item Sensor LiDAR \$4,000
            \item Brazo robótico \$3,700
            \item Sensor multiespectral \$3,500
        \end{itemize}
        
    \end{center}
    \item Determinar si existen economías de escala y cómo afectarían la estrategia
    de precios.
    \begin{center}
        Economías a escala es reducir el costo promedio por unidad para hacer más barata la creación de dicha unidad. 
        \vspace{0.3cm}\\ 
        Costo fijo: \$150,000 \\ 
        Cálculo del costo por unidad:
        \vspace{0.3cm}\\ 
        \hrule 
        \textbf{10 unidades }
        \begin{itemize}
            \centering
            \item Camara: \$3,300 * 10 = \$33,000
            \item Sensor LiDAR \$4,000 * 10 = \$40,000
            \item Brazo robótico \$3,700 * 10 =\$37,000
            \item Sensor multiespectral \$3,500 * 10 = \$35,000
        \end{itemize}
        Costos fijos por unidad: $\displaystyle\frac{\$150,000}{10} = 15,000$
        \vspace{0.3cm}\\ 
        Costo total por unidad:
        \begin{itemize}
            \centering
            \item Camara: \$3,300 + \$15,000 = \$18,300
            \item Sensor LiDAR \$4,000 + \$15,000 = \$19,000
            \item Brazo robótico \$3,700 + \$15,000 = \$18,700
            \item Sensor multiespectral \$3,500 + \$15,000 = \$18,500
        \end{itemize} 
        Restando el precio de venta por costo total por unidad, se presentan \textbf{grandes pérdidas}.
        \hrule
        \textbf{50 unidades }
            \begin{itemize}
                \centering
                \item Camara: \$3,300 * 50 = \$165,000
                \item Sensor LiDAR \$4,000 * 50 = \$200,000
                \item Brazo robótico \$3,700 * 50 =\$185,000
                \item Sensor multiespectral \$3,500 * 50 = \$175,000
            \end{itemize}
            Costos fijos por unidad: $\displaystyle\frac{\$150,000}{50} = \$3,000$
            \vspace{0.3cm}\\ 
            Costo total por unidad:
            \begin{itemize}
                \centering
                \item Camara: \$3,300 + \$3,000 = \$6,300
                \item Sensor LiDAR \$4,000 + \$3,000 = \$7,000
                \item Brazo robótico \$3,700 + \$3,000 = \$6,700
                \item Sensor multiespectral \$3,500 + \$3,000 = \$6,500
            \end{itemize} 
            Restando el precio de venta por costo total por unidad, se presentan \textbf{grandes pérdidas}.
            \hrule
            \textbf{100 unidades }
            \begin{itemize}
                \centering
                \item Camara: \$3,300 * 100 = \$330,000
                \item Sensor LiDAR \$4,000 * 100 = \$400,000
                \item Brazo robótico \$3,700 * 100 =\$370,000
                \item Sensor multiespectral \$3,500 * 100 = \$350,000
            \end{itemize}
            Costos fijos por unidad: $\displaystyle\frac{\$150,000}{100} = \$1,500$
            \vspace{0.3cm}\\ 
            Costo total por unidad:
            \begin{itemize}
                \centering
                \item Camara: \$3,300 + \$1,500 = \$4,800
                \item Sensor LiDAR \$4,000 + \$1,500 = \$5,500
                \item Brazo robótico \$3,700 + \$1,500 = \$5,200
                \item Sensor multiespectral \$3,500 + \$1,500 = \$5,000
            \end{itemize} 
            Restando el precio de venta por costo total por unidad, se presentan \textbf{más ganancias}. \\ Solo con pedidos grandes se empieza a obtener alguna rentabilidad con ciertos módulos
    \end{center}

\end{itemize}
\newpage
\textit{Análisis de Make-or-Buy:}
\begin{itemize}
    \item Evaluar si es más rentable fabricar los módulos internamente o
    subcontratarlos.
    \begin{itemize}
        \item Camara: \$3,300
        \item Sensor LiDAR \$4,000
        \item Brazo robótico \$3,700
        \item Sensor multiespectral \$3,500
    \end{itemize}
    \item Considerar factores cuantitativos y cualitativos en la decisión.
\end{itemize}
\textit{Presupuesto de capital:}
\begin{itemize}
    \item Calcular el costo de capital para DroneTech si deciden expandir su
    capacidad de producción.
    \item Evaluar diferentes opciones de financiamiento y su impacto en los costos
    totales.
\end{itemize}

\subsection*{Preguntas de análisis}
\begin{enumerate}
    \item ¿Cómo afectaría la decisión de pricing la estructura de costos de
    DroneTech?.
    \item ¿Qué estrategias podría implementar DroneTech para reducir sus costos sin
    comprometer la calidad?
    \item ¿Cómo podrían los avances tecnológicos futuros afectar la estructura de costos de VersiDron?
    \item ¿Qué consideraciones éticas deben tenerse en cuenta al analizar los costos
    y determinar los precios?
    \item ¿Cómo podría DroneTech utilizar el análisis de costos para obtener una
    ventaja competitiva en el mercado de drones?
\end{enumerate}


\end{sloppypar}
\end{document}