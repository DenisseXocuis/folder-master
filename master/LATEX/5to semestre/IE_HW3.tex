\documentclass[letterpaper,12pt]{article}
\usepackage[utf8]{inputenc}
\usepackage{charter}
\usepackage{geometry}
\usepackage{amsmath}
\usepackage{float}
\usepackage{graphicx}
\usepackage{subcaption}
\usepackage{amssymb}
\usepackage{adjustbox}
\usepackage{wrapfig} 
\usepackage{xcolor}
\usepackage{fancyhdr}
\usepackage{tabularx}
\usepackage{fancyhdr}
\usepackage{comment}

\geometry{top=2cm, bottom=2cm, left=2cm, right= 2cm} %%margen
\graphicspath{{images/}}
\parindent=0pt

\begin{document}
%%para contador y eso
\thispagestyle{empty}
\newpage
\setcounter{page}{1}
\pagestyle{headings}
%%%%%%%%%%%%%%%%%%%%%%%%%%%%%%%%%%%%%%%%%%%%%%%%%%%%%%%%%%%%%%%%%%%%%%%%%%%%
\begin{sloppypar} 
    %%%portadita
    \begin{titlepage}
        \fancyhead{R}{
        \includegraphics[width=0.1\linewidth]{logoUV.png}
        }
        \hspace{2.5cm}
        {\bfseries\LARGE Universidad Veracruzana \par}
        \hspace{2cm}
        {\scshape\Large Facultad de Ingeniería Eléctrica y Electrónica \par}
        \begin{center}
            \vspace{7cm}
            {\itshape\huge Informe de Análisis de Costos \par}
            {\large Lara Xocuis Martha Denisse\par}
            {\large S22002213 \par}
            \vfill
            {\Large September 16, 2024 \par}
        \end{center}
    \end{titlepage} 

%%%%%%%%%%%%%%%%%%%%%%main template%%%%%%%%%%%%%%%%%%%%%%
\section*{Introducción}
Este informe examina diversos conceptos de costes relacionados con la tecnología de los drones, centrándose en el análisis coste-cantidad-utilidad, el cálculo de costes basado en actividades (ABC), el análisis de costes marginales, la estimación de la producción o la adquisición, y el presupuesto de capital.
\vspace{0.3cm}\\ 
El objetivo es proporcionar una comprensión clara de cómo afectan estos análisis a la toma de decisiones estratégicas y a la competitividad de una empresa en el mercado.
\newpage
\section*{EJERCICIOS DE INGENIERÍA ECONÓMICA ENTREGA 03}
\subsection*{INSTRUCCIONES: Responde a las preguntas del siguiente CASO de estudio}
\subsection{Análisis de Costos en el Desarrollo de un Dron Multifuncional}

%%%%%%%%%%%%%%%%%%info y así%%%%%%%%%%%%%%%%%%%
\textbf{\textit{CONTEXTO:}}
\begin{figure}[H]
    \centering 
    \includegraphics[width=0.7\textwidth]{dronetech.png}
\end{figure}
DroneTech es una startup fundada por tres recién graduados: Elena (Ing. Electrónica), Marco (Ing. Mecatrónica) y Sofía (Ing. Informática). La empresa se especializa en el desarrollo de drones para aplicaciones industriales y de investigación. Después de un año de I+D, han creado un prototipo de dron
multifuncional llamado "VersiDron", diseñado para adaptarse a diversas aplicaciones cambiando sus módulos.
\vspace{0.3cm}\\ 
\textbf{\textit{OBJETIVO DEL CASO:}}
\textit{Aplicar diversos conceptos de costo en un escenario realista que combina aspectos de ingeniería electrónica, mecatrónica e informática en el contexto del desarrollo y producción de drones.}
\vspace{0.3cm}\\ 
\textbf{\textit{ACTIVIDADES A REALIZAR:}}
\vspace{0.3cm}\\ 
\textit{Clasificación de costos:}
\begin{itemize}
    \item Identificar y clasificar los diferentes tipos de costos (fijos, variables, directos, indirectos, hundidos, etc.)
    \begin{itemize}
        \item \textbf{Fijos:} Gastos generales y administrativos y costo de instalaciones de producción
        \item \textbf{Variables:} Componentes de estructura base, costos de módulos intercambiables y mano de obra para ensamblaje.
        \item \textbf{Directos:} Costo de modulos, mano de obra por ensamblaje
        \item \textbf{Indirectos}: Marketing y ventas
        \item \textbf{Hundidos}: Desarollo de software
    \end{itemize}
    \newpage
    \item Explicar cómo cada tipo de costo afecta la toma de decisiones en la producción de VersiDron.\\ Los costos fijos e indirectos son fundamentales para comprender la estructura general de costos y la rentabilidad global, los costos variables y directos afectan directamente a las decisiones sobre volúmenes de producción, selección de unidades y fijación de precios. Para fijar precios competitivos y cubrir todos los costos relevantes, hay que tener en cuenta los costos de marketing y ventas.
\end{itemize}
\textit{Análisis de Costo-Volumen-Utilidad:}
\begin{itemize}
    \item Calcular el punto de equilibrio para cada configuración de VersiDron (con
    diferentes módulos).
    \begin{center}
    Suma de costos fijos:
    $\$100,000 + \$50,000 = \$150,000$ \\ 
    Suma de costos variables por cada modulo: $\$2,000 + (modulo) + \$500$ \\ 
    Y suponiendo que el precio de venta es de 5mil pesos:
    \begin{itemize}
        \item Camara: \$2,000 + \$800 + \$500 = \$3,300 \\ Punto de equilibrio:
        
        $\displaystyle \frac{\$150,000}{\$5,000 - \$3,300} = 88$ unidades
        \item Sensor LiDAR:  \$2,000 + \$1,500 + \$500 = \$4,000 \\ Punto de equilibrio:
        
        $\displaystyle \frac{\$150,000}{\$5,000 - \$4,300} = 150$ unidades
        \item Brazo robótico:  \$2,000 + \$1,200 + \$500 = \$3,700 \\ Punto de equilibrio:
        
        $\displaystyle \frac{\$150,000}{\$5,000 - \$3,700} = 115$ unidades
        \item Sensor multiespectral:  \$2,000 + \$1,000 + \$500 = \$3,500 \\ Punto de equilibrio:
        
        $\displaystyle \frac{\$150,000}{\$5,000 - \$3,500} = 100$ unidades
    \end{itemize}
    \end{center} 
    \item Determinar cómo cambia el punto de equilibrio si se consideran los tres pedidos juntos. 
    \begin{center}
    Pedidos totales: 50 + 10 + 100 = 160 unidades
    \begin{itemize}
        \item Camara: \$3,300 * 160 = \$528,000
        \item Sensor LiDAR \$4,000 * 160 = \$640,000
        \item Brazo robótico \$3,700 * 160 = \$592,000
        \item Sensor multiespectral \$3,500 * 160 = \$560,000
    \end{itemize}
    \end{center}
\end{itemize}
\newpage
\textit{Costeo Basado en Actividades (ABC):}
\begin{itemize}
    \item Desarrollar un sistema de costeo ABC para la producción de VersiDron. \\  Diseñar, adquirir materiales, montar, probar, empacar, marketing, y administración, factores que causan los costos, como horas de desarrollo, órdenes de compra, unidades ensambladas, etc.
    \item Comparar los resultados del costeo ABC con un enfoque de costeo tradicional.
    \begin{center}
        \begin{tabular}[H]{|c|c|} \hline 
            ABC & Tradicional \\ \hline
            Visión precisa de los costos & Asigna costos generales\\ 
                & y fijos de manera uniforme\\ \hline
            Mejor comprensión & Menos exacto\\ \hline
        \end{tabular}
    \end{center}
\end{itemize}
\textit{Análisis de Costos Marginales:}
\begin{itemize}
    \item Calcular el costo marginal de producir unidades adicionales para cada
    cliente.
    \begin{center}

        \begin{itemize}
            \item Camara: \$3,300
            \item Sensor LiDAR \$4,000
            \item Brazo robótico \$3,700
            \item Sensor multiespectral \$3,500
        \end{itemize}
        
    \end{center}
    \item Determinar si existen economías de escala y cómo afectarían la estrategia
    de precios.
    \begin{center}
        Economías a escala es reducir el costo promedio por unidad para hacer más barata la creación de dicha unidad. 
        \vspace{0.3cm}\\ 
        Costo fijo: \$150,000 \\ 
        Cálculo del costo por unidad:
        \vspace{0.3cm}\\ 
        \hrule 
        \textbf{10 unidades }
        \begin{itemize}
            \centering
            \item Camara: \$3,300 * 10 = \$33,000
            \item Sensor LiDAR \$4,000 * 10 = \$40,000
            \item Brazo robótico \$3,700 * 10 =\$37,000
            \item Sensor multiespectral \$3,500 * 10 = \$35,000
        \end{itemize}
        Costos fijos por unidad: $\displaystyle\frac{\$150,000}{10} = 15,000$
        \vspace{0.3cm}\\ 
        Costo total por unidad:
        \begin{itemize}
            \centering
            \item Camara: \$3,300 + \$15,000 = \$18,300
            \item Sensor LiDAR \$4,000 + \$15,000 = \$19,000
            \item Brazo robótico \$3,700 + \$15,000 = \$18,700
            \item Sensor multiespectral \$3,500 + \$15,000 = \$18,500
        \end{itemize} 
        Restando el precio de venta por costo total por unidad, se presentan \textbf{grandes pérdidas}.
        \hrule
        \textbf{50 unidades }
            \begin{itemize}
                \centering
                \item Camara: \$3,300 * 50 = \$165,000
                \item Sensor LiDAR \$4,000 * 50 = \$200,000
                \item Brazo robótico \$3,700 * 50 =\$185,000
                \item Sensor multiespectral \$3,500 * 50 = \$175,000
            \end{itemize}
            Costos fijos por unidad: $\displaystyle\frac{\$150,000}{50} = \$3,000$
            \vspace{0.3cm}\\ 
            Costo total por unidad:
            \begin{itemize}
                \centering
                \item Camara: \$3,300 + \$3,000 = \$6,300
                \item Sensor LiDAR \$4,000 + \$3,000 = \$7,000
                \item Brazo robótico \$3,700 + \$3,000 = \$6,700
                \item Sensor multiespectral \$3,500 + \$3,000 = \$6,500
            \end{itemize} 
            Restando el precio de venta por costo total por unidad, se presentan \textbf{grandes pérdidas}.
            \hrule
            \textbf{100 unidades }
            \begin{itemize}
                \centering
                \item Camara: \$3,300 * 100 = \$330,000
                \item Sensor LiDAR \$4,000 * 100 = \$400,000
                \item Brazo robótico \$3,700 * 100 =\$370,000
                \item Sensor multiespectral \$3,500 * 100 = \$350,000
            \end{itemize}
            Costos fijos por unidad: $\displaystyle\frac{\$150,000}{100} = \$1,500$
            \vspace{0.3cm}\\ 
            Costo total por unidad:
            \begin{itemize}
                \centering
                \item Camara: \$3,300 + \$1,500 = \$4,800
                \item Sensor LiDAR \$4,000 + \$1,500 = \$5,500
                \item Brazo robótico \$3,700 + \$1,500 = \$5,200
                \item Sensor multiespectral \$3,500 + \$1,500 = \$5,000
            \end{itemize} 
            Restando el precio de venta por costo total por unidad, se presentan \textbf{más ganancias}. \\ Solo con pedidos grandes se empieza a obtener alguna rentabilidad con ciertos módulos
    \end{center}

\end{itemize}
\newpage
\textit{Análisis de Make-or-Buy:}
\begin{itemize}
    \item Evaluar si es más rentable fabricar los módulos internamente o
    subcontratarlos.
    \begin{itemize}
        \item Camara: \$3,300
        \item Sensor LiDAR \$4,000
        \item Brazo robótico \$3,700
        \item Sensor multiespectral \$3,500
    \end{itemize}
    Total: \$800 + \$1,500 + \$1,200 + \$1,000 + \$500 = \$5,000 por módulo
    \vspace{0.3cm}\\ 
    Si el coste es el mismo, la decisión debe basarse en otros factores, como el control de calidad, la flexibilidad y la capacidad de respuesta al cambio. Si su tecnología de drones puede garantizar la calidad y la flexibilidad, la fabricación interna puede ser más ventajosa. Sin embargo, si lo que busca es ampliar rápidamente y reducir la carga de trabajo, la subcontratación puede ser una mejor opción.
    \item Considerar factores cuantitativos y cualitativos en la decisión.
    \begin{itemize}
        \item Cuantitativos: fabricación Interna $\rightarrow$ \$5,000 por módulo; Gastos Generales y Administrativos $\rightarrow$ \$100,000 anuales.
        \item Cualitativos: Mayor control sobre la calidad del producto; Desarrollar habilidades y conocimiento en el equipo
    \end{itemize}
    Es importante tener en cuenta no sólo los costes directos, sino también el impacto a largo plazo en la calidad, la flexibilidad y la innovación. Si la diferenciación y la personalización son importantes, la fabricación interna puede ser la mejor opción. Si la eficiencia y la rapidez de comercialización son importantes, la externalización puede ser más ventajosa.
\end{itemize}
\textit{Presupuesto de capital:}
\begin{itemize}
    \item Calcular el costo de capital para DroneTech si deciden expandir su
    capacidad de producción.
    \begin{itemize}
        \item Deuda: tasa de interés 6\%; ajustado por impuestos 30\%
        $$6\% * (1-0.30) = 4.2\%$$
        \item Capital propio: 9\%
        \item Proporciones: capital propio 71.4\%; deuda 28.6\%
    \end{itemize}
    $$= (71.4\% * 9\%) + (28.6\% * 4.2\%) \approx  6.25\%$$
    \item Evaluar diferentes opciones de financiamiento y su impacto en los costos
    totales.
    \vspace{0.3cm}\\ 
    La elección del método de financiación depende de la situación financiera actual de su empresa, su tolerancia al riesgo y sus objetivos a largo plazo.\\
    Si se quiere minimizar la dilución del capital y conservar el control de la empresa, el capital de deuda puede ser adecuado.
    Si la empresa quiere evitar el riesgo de reembolso y no preocuparse por la dilución del capital, la financiación mediante capital propio puede ser más adecuada.
    Si necesita flexibilidad y bajos costes iniciales, el arrendamiento financiero puede ser ideal.
\end{itemize}

\subsection*{Preguntas de análisis}
\begin{enumerate}
    \item \textbf{¿Cómo afectaría la decisión de pricing la estructura de costos de
    DroneTech?.}\\ 
    La fijación de precios es fundamental para DroneTech, ya que afecta a su ROI, rentabilidad y competitividad en el mercado. 
    \vspace{0.3cm}\\ 
    La empresa debe evaluar sus costes y su propuesta de valor para desarrollar una estrategia de precios adecuada que garantice la sostenibilidad y el crecimiento.
    \item \textbf{¿Qué estrategias podría implementar DroneTech para reducir sus costos sin
    comprometer la calidad?} \\ Es posible reducir costes sin comprometer la calidad optimizando la cadena de suministro, negociando mejores condiciones con los proveedores y comprando más productos. También son importantes un diseño del producto que facilite la producción y una gestión eficaz del inventario. Los costes de compra también pueden reducirse mediante el marketing digital.
    \item \textbf{¿Cómo podrían los avances tecnológicos futuros afectar la estructura de costos de VersiDron? }\\ 
    Los avances tecnológicos pueden reducir significativamente los costes de producción y explotación, aumentar la eficiencia, ofrecer nuevas oportunidades de innovación y aumentar la competitividad de VersiDron en el mercado. 
    \item \textbf{¿Qué consideraciones éticas deben tenerse en cuenta al analizar los costos
    y determinar los precios?}
    \begin{itemize}
        \item Impacto Social: Evaluar cómo los precios afectan a diferentes segmentos de la población.
        \item Calidad del Producto: Mantener la calidad sin comprometerla para reducir costos.
        \item Competencia Leal: No participar en prácticas anticompetitivas que dañen el mercado.
        \item Transparencia: Comunicar claramente cómo se determinan los precios.
    \end{itemize}
    \item \textbf{¿Cómo podría DroneTech utilizar el análisis de costos para obtener una
    ventaja competitiva en el mercado de drones?}
    \begin{itemize}
        \item Optimización de Producción: Identificar y eliminar ineficiencias para reducir gastos y mejorar márgenes.
        \item Mejora de Rentabilidad: Controlar costos fijos y variables para ajustar estrategias y mantener márgenes saludables.
        \item Inversiones: Evaluar proyectos para asegurar retornos positivos y reducir costos futuros.
    \end{itemize}
\end{enumerate}

\end{sloppypar}
\end{document}