\documentclass[letterpaper,12pt]{article}
\usepackage[utf8]{inputenc}
\usepackage{charter}
\usepackage{geometry}
\usepackage{amsmath}
\usepackage{float}
\usepackage{graphicx}
\usepackage{subcaption}
\usepackage{amssymb}
\usepackage{adjustbox}
\usepackage{wrapfig} 
\usepackage{xcolor}
\usepackage{fancyhdr}
\usepackage{tabularx}
\usepackage{fancyhdr}
\usepackage{comment}

\geometry{top=2cm, bottom=2cm, left=2cm, right= 2cm} %%margen
\graphicspath{{images/}}
\parindent=0pt

\begin{document}
%%para contador y eso
\thispagestyle{empty}
\newpage
\setcounter{page}{1}
\pagestyle{headings}
%%%%%%%%%%%%%%%%%%%%%%%%%%%%%%%%%%%%%%%%%%%%%%%%%%%%%%%%%%%%%%%%%%%%%%%%%%%%
\begin{sloppypar} 
    %%%portadita
    \begin{titlepage}
        \fancyhead{R}{
        \includegraphics[width=0.1\linewidth]{logoUV.png}
        }
        \hspace{2.5cm}
        {\bfseries\LARGE Universidad Veracruzana \par}
        \hspace{2cm}
        {\scshape\Large Facultad de Ingeniería Eléctrica y Electrónica \par}
        \begin{center}
            \vspace{7cm}
            {\itshape\huge Bases de Datos \par}
            {\large Lara Xocuis Martha Denisse\par}
            {\large S22002213 \par}
            \vfill
            {\Large \today \par}
        \end{center}
    \end{titlepage} 

%%%%%%%%%%%%%%%%%%%%%%INFO%%%%%%%%%%%%%%%%%%%%%%
\textbf{ACTIVIDADES:}
\section*{Leer el capítulo 1 del libro: "FUNDAMENTOS DE BASES DE DATOS" de Abraham Silberschatz}
\begin{itemize}
    \item \textbf{¿Cuál es el propósito de los sistemas de bases de datos?} \\ Uno de los propósitos principales de los sistemas de bases de datos es ofrecer a los usuarios una visión abstracta de los datos. Es decir, buscan acabar con la redundancia e inconsistencia de los datos, dificultad, aislamiento, problemas de integridad y seguridad.
    \item \textbf{¿Cuáles son las desventajas del sistema de procesamiento de archivos?} \\ Dependencia de datos del programa, duplicación de datos, compartir datos de forma limitada, tiempos de desarrollo largos y mantenimiento excesivo del programa.
    \item \textbf{Analice con detalle los tres niveles de abstracción de datos}
    \begin{enumerate}
        \item Nivel \textbf{FÍSICO}: Describe \textbf{\textit{cómo}} se almacenan realmente los datos.
        \item Nivel \textbf{LÓGICO}: Describe \textbf{\textit{qué}} datos se almacenan en la base de datos y que relaciones existen entre esos datos.
        \item Nivel \textbf{VISTA}: Simplifica su \textbf{interacción} con el sistema.
    \end{enumerate}
    \item \textbf{Revisar la sección 1.11 Arquitectura de las bases de datos} \\ 
    La arquitectura de los sistemas de bases de datos puede variar considerablemente según el entorno en el que se implementen. Existen diferentes tipos de arquitecturas:
    \begin{itemize}
        \item \textbf{Centralizadas}: Donde la base de datos se gestiona desde un único sistema central.
        \item \textbf{Cliente-servidor}: En esta configuración, una máquina servidora maneja las solicitudes de múltiples clientes.
        \item \textbf{Paralelas}: Aprovechan la computación paralela para procesar consultas y datos de manera eficiente.
        \item \textbf{Distribuidas}: Se extienden a través de múltiples máquinas localizadas en distintas ubicaciones geográficas.
    \end{itemize}

    \item \textbf{Analice con detalle la figura 1.7 Arquitecturas de dos y tres capas. }\\ \textit{\underline{Arquitectura de dos capas:}} La aplicación se divide en un componente cliente, que se comunica directamente con el servidor de bases de datos a través de lenguajes de consulta como SQL, utilizando estándares como ODBC y JDBC. 
    \vspace{0.3cm}\\ 
    \textit{\underline{Arquitectura de tres capas:}} El cliente actúa solo como interfaz para el usuario. La comunicación entre el cliente y el servidor de bases de datos se realiza a través de un servidor de aplicaciones que gestiona la lógica de negocio y las interacciones con la base de datos. Esta arquitectura es adecuada para aplicaciones grandes y en la web.
    \newpage
    \item \textbf{Leer la sección 1.12.1 Usuarios de bases de datos e interfaces de usuario.}
    \begin{itemize}
        \item \textbf{Usuarios normales}: Usuarios no sofisticados que interactuan con el sistema invocando alguno de los programas de aplicación que se han escrito previamente.
        \item \textbf{Programadores de aplicaciones}: profesionales informáticos que escriben programas de aplicación.
        \item \textbf{Usuarios sofisticados}: interactuan con el sistema sin escribir programas, formulan sus consultas en un lenguaje de consultas de bases de datos.
        \item \textbf{Usuarios especializados}: escriben aplicaciones de bases de datos especializadas que no encajan en el marco tradicional del procesamiento de datos.
    \end{itemize}


    \item \textbf{Estudiar la sección 1.13 Historia de los sistemas de bases de datos:} 
    \begin{enumerate}
        \item Principios del siglo XX: Las tarjetas perforadas, inventadas por Herman Hollerith, se usaron para el censo de EE.UU. y se introdujeron en las computadoras para registrar datos.
        \item Década de 1950 y principios de 1960: Se desarrollaron cintas magnéticas para almacenar datos, permitiendo la automatización de tareas como la elaboración de nóminas.
        \item Finales de los años 60 y 70: La introducción de discos duros permitió el acceso directo a los datos, liberando el procesamiento de la secuencialidad. 
        \item Años 80: A pesar de la complejidad inicial, las bases de datos relacionales superaron a las de red y jerárquicas en rendimiento y facilidad de uso, gracias a proyectos como System R de IBM
        \item Principios de los años 90: Se enfatizó en el uso de SQL para aplicaciones intensivas en consultas y análisis de datos.
        \item Principios del siglo XXI: Se introdujo XML y su lenguaje de consulta, XQuery, en las bases de datos. 
    \end{enumerate}
    \item \textbf{Responder de la sección de Ejercicios prácticos}
    \begin{itemize}
        \item 1.1 En este capítulo se han descrito varias ventajas importantes de los sitemas gestores de bases de datos. ¿Cuáles son sus dos inconvenientes?
        \item 1.2 Indíquese siete lenguajes de programación que sean procedimentales y dos que no lo sean. ¿Qué grupo es más fácil de aprender a usar? Explíquese la respuesta.
        \item 1.5 Indíquese cuatro aplicaciones que se haya usado que sea muy posible que utilicen un sistema de bases de datos para almacenar datos persistentes.
        \item 1.6 Indíquense cuatro diferencias significativas entre un sistema de procesamiento de archivos y un SGBD.
        \item 1.7 Explíquese la diferencia entre independencia de datos física y lógica.
    \end{itemize}
\end{itemize}
%%%%%%%%%%%%%%%%%%%%%%%%%%%%%%%%%%%%%555
\section*{Leer la Parte 1 del libro: "MODERN DATABASE MANAGEMENT" de Jeffrey A. Hoffer: "The Context of Database Management"}
\begin{itemize}
    \item \textbf{Lea con cuidado la sección "BASIC CONCEPTS AND DEFINITIONS" }
    \begin{itemize}
        \item \textbf{Note la diferencia entre dato e información}
        \begin{center}
                \begin{tabular}[H]{|c|c|} \hline 
                \textbf{DATO} & \textbf{INFORMACIÓN} \\ \hline  
                - Representación almacenada de & - Datos que han sido procesados\\
                objetos y eventos & que permiten que el conocimiento \\
                que tienen un significado & del usuario incremente \\
                importante para el usuario & \\ \hline
                \end{tabular}   
        \end{center}

        \item \textbf{Ponga especial énfasis en “Metadatos”} \\ Comúnmente es conocido como "\textit{datos de datos}", describe las \textbf{características} de los datos. 
        \item \textbf{¿Cuáles son las desventajas del sistema de procesamiento de archivos?}
        \begin{itemize}
            \item \textbf{Duplicación} de datos
            \item \textbf{Dependencia} de programas.
            \item Compartir datos de forma \textbf{limitada}
            \item \textbf{Tiempo extenso} de desarrollo
            \item \textbf{Mantenimiento excesivo} del programa
        \end{itemize}
        \item \textbf{Analice la definición de este autor para un SGBD} \\ Un Sistema Gestor de Base de Datos es un software que es usado para crear, manipular y controlar el acceso a los usuarios a una base de datos. Provee un método sistemático para \textbf{crear, actualizar y almacenar datos} guardados en una BD.
        \item \textbf{¿Cuáles son las ventajas/desventajas de la propuesta de bases de datos?}
        \begin{center}
            \begin{tabular}[H]{|c|c|} \hline
            \textbf{VENTAJAS} & \textbf{DESVENTAJAS} \\ \hline
            - Independencia hacia un programa & -Personal nuevo y especializado\\ \hline 
            - Redundancia mínima & -Instalación y mantenimiento costoso\\ \hline
            - Mejora la consistencia de los datos & -Necesidad por copias de respaldo explicitas \\ \hline
            - Mejora la accesibilidad de datos & - Conflicto empresarial \\ \hline
            - Mejora la calidad de datos & \\ \hline
                
            \end{tabular}
        \end{center}

        \item \textbf{¿Qué es una restricción?}\\ Una regla que no puede ser violada por el usuario de la base de datos
        \item \textbf{¿Cuáles son los costos y riesgos de utilizar BD?}
        \begin{itemize}
            \item Personal nuevo y especializado
            \item Instalación y mantenimiento costoso
            \item Necesidad por copias de respaldo explícitas
            \item Conflicto empresarial
        \end{itemize}
        \newpage
        \item \textbf{¿Cuáles son los componentes del ambiente de BD?}
        \begin{itemize}
            \item CASE TOOLS: Software de ingeniería.
            \item Repositorios: almacenamiento de \textbf{metadatos} centralizados
            \item Base de datos: almacén de datos
            \item Programas: software que usan los datos
            \item Interfaz de usuario: pantalla textual y gráfica para los usuarios.
            \item Administrador de datos: personal responsable para el manejo de una base de datos
            \item Desarolladores de sistemas
            \item Programadores: personal responsable para diseñar una base de datos y software
            \item "END-USERS": personas quienes usan las aplicaciones
        \end{itemize}
        \item \textbf{Analice el ambiente de aplicación de las BD de dos y tres capas}  
        \item \textbf{¿Qué es un ERP?} \\ Sistemas que han sido envueltos desde el material
    \end{itemize}
    \item Revise la sección correspondiente a la evolución de los sistemas de BD
    \item Revisar el ciclo de vida de desarrollo de sistemas, que también aplica a las BD
    \item Analizar con detalle el resumen del capitulo
    \item Realizar los ejercicios de la página 44 hasta la pregunta 12
\end{itemize}
\section*{Leer el capítulo 1 del libro: "FUNDAMENTOS DE BASES DE DATOS" de Martha Elena Millán}
\begin{itemize}
    \item Analizar con detalle la arquitectura ANSI-SPARC de un SMBD
\end{itemize}


\end{sloppypar}
\end{document}