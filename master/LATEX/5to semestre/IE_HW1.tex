\documentclass[letterpaper,12pt]{article}
\usepackage[utf8]{inputenc}
\usepackage{charter}
\usepackage{geometry}
\usepackage{amsmath}
\usepackage{float}
\usepackage{graphicx}
\usepackage{subcaption}
\usepackage{amssymb}
\usepackage{adjustbox}
\usepackage{wrapfig} 
\usepackage{xcolor}
\usepackage{fancyhdr}
\usepackage{tabularx}
\usepackage{fancyhdr}
\usepackage{comment}

\geometry{top=2cm, bottom=2cm, left=2cm, right= 2cm} %%margen
\graphicspath{{images/}}
\parindent=0pt

\begin{document}
%%para contador y eso
\thispagestyle{empty}
\newpage
\setcounter{page}{1}
\pagestyle{headings}
%%%%%%%%%%%%%%%%%%%%%%%%%%%%%%%%%%%%%%%%%%%%%%%%%%%%%%%%%%%%%%%%%%%%%%%%%%%%
\begin{sloppypar} 
    %%%portadita
    \begin{titlepage}
        \fancyhead{R}{
        \includegraphics[width=0.1\linewidth]{logoUV.png}
        }
        \hspace{2.5cm}
        {\bfseries\LARGE Universidad Veracruzana \par}
        \hspace{2cm}
        {\scshape\Large Facultad de Ingeniería Eléctrica y Electrónica \par}
        \begin{center}
            \vspace{7cm}
            {\itshape\huge INGENIERÍA ECONÓMICA \par}
            {\large Lara Xocuis Martha Denisse\par}
            {\large S22002213 \par}
            \vfill
            {\Large \today \par}
        \end{center}
    \end{titlepage} 

%%%%%%%%%%%%%%%%%%%%%%INFO%%%%%%%%%%%%%%%%%%%%%%
\section*{EJERCICIOS DE INGENIERÍA ECONÓMICA ENTREGA 01}
\subsection*{INSTRUCCIONES: Responde a las preguntas del siguiente CASO de estudio}
\subsection{Innovación en Sistemas de Pago}
\textbf{Preguntas de análisis:}
\begin{itemize}
    \item ¿Cómo podría este sistema de pago afectar el concepto tradicional de dinero y transacciones financieras?\\ Adaptación al mundo actual respecto al avance tecnológico y formas de manejo de la economía de cada país, es decir, esta gran potencia tecnológica implicaría la privacidad del cliente, salud, mayor compromiso de mantenimiento y, de esto, más capital para que el sistema sea estable.
    \item ¿Qué desafíos éticos y de privacidad plantea la implementación de un chip implantable para transacciones financieras? \\ El cliente no podría sentirse seguro con el chip (cabe la posibilidad de obtener más que su ADN), podría ser riesgoso para la salud y exposición a gran historial tecnológico.
    \item Desde una perspectiva de ingeniería, ¿cuáles son los principales desafíos técnicos que deben superarse para que este sistema sea viable y seguro? \\ Buen manejo de la base de datos del sistema, mantenimiento y covertura, alto rendimiento global de privacidad y compromiso con la salud del cliente, gran desarrollo de tecnología autosuficiente.
    \item ¿Cómo podría este sistema impactar en la economía global y en el papel de las instituciones financieras tradicionales? \\ Queda como un negocio a competencia y abre paso a nuevas potencias económicas, de alguna forma y si inicia con un gran éxito, podría causar una inestabilidad global con la capital del país.
    \item Considerando las implicaciones sociales y económicas, ¿cómo podría TechPay Solutions abordar la adopción gradual de esta tecnología para minimizar la disrupción en los sistemas financieros existentes? \\ Aplicar estos nuevos recursos poco a poco sin ser la única forma vial de manejo de pago, mantenerlo como opción mientras tal vez mostrar el sistema a otros distintos bancos para, al mismo tiempo y de forma gradual, todo el sistema bancario vaya transformandose a la par.
\end{itemize}
\newpage
\subsection{Financiamiento de Proyecto de Automatización en Industria 4.0}
\textbf{Preguntas de análisis:}
\begin{itemize}
    \item ¿Cuál será el monto total que TechnoFuture deberá pagar al final del período de 3 años si acepta el préstamo de InnoFinance? \\ Para el interés: 500,000 * .08 * 3 = 120,000 pesos \\ Para el monto total: 500,000 + 120 000 = 620,000
    \item Si TechnoFuture proyecta que sus ingresos netos anuales serán de \$250,000 una vez que el producto esté en el mercado, ¿cuánto tiempo les tomará pagar el préstamo si dedican el 50\% de sus ingresos netos anuales al pago de la deuda? \\ 250,000 * .5 = 125,000 \\ Tiempo que les tomará pagar = $\frac{500,000}{125,000}$ = 4 años
    \item Considerando que el desarrollo del producto tomará 1 año antes de generar ingresos, ¿cómo afectaría esto al costo total del préstamo y a la capacidad de TechnoFuture para pagarlo? \\ El dinero estaría ahí en espera hasta que hayan ingresos y no se podrán generar pagos, los intereses se van a ir acumulando.
    \item Si un competidor ofrece a TechnoFuture una inversión de \$400,000 a cambio del 25\% de la empresa, ¿cómo se compara esta oferta con el préstamo bancario en términos de costo financiero a largo plazo? \\ Si la empresa tiene mayor éxito, puede que pierda mucho valor capital si se acepta la propuesta
    \item Desde una perspectiva técnica, ¿cómo podrían los ingenieros de TechnoFuture optimizar el desarrollo del producto para reducir costos y tiempo, minimizando así la necesidad de financiamiento externo? \\ Podrían estar al tanto de las nuevas tecnologías y su novedad para aprovechar distintos recursos que tengan a la mano y así, reducir costos.
\end{itemize}


\end{sloppypar}
\end{document}