\documentclass[a4paper,12pt]{article}
\usepackage[utf8]{inputenc}
\usepackage{geometry}
\usepackage{float}
\usepackage{graphicx}
\usepackage{subcaption}
\usepackage{amssymb}
\usepackage{adjustbox}
\usepackage{wrapfig} %%imagen envuelta por un texto
\usepackage{xcolor}
\usepackage{fancyhdr}

\title {\textbf{Investigación}}
\author{}
\date{}
\geometry{top=2cm, bottom=2cm, left=2cm, right= 2cm} %%margen

\begin{document}
\maketitle
\section{Osciloscopio}
Es un instrumento de medición de visualización de gráficos que permite ver ondas y oscilasciones, en donde se representa gráficamente la magnitud de una señal eléctrica. En la mayoría de las aplicaciones, el gráfico muestra cómo cambian las señales con el tiempo.

Existen fuerzas o señales que se mueven en forma de onda como olas de mar, todas esas fuerzas son invisibles al ojo del ser humano, así que es necesario un equipo que nos permita poder verlos y el osciloscopio funciona para ello.

\textbf{¿Qué más puede hacer el osciloscopio?} Al medir los cambios de las ondas a través del tiempo, pueden dar mediciones súper importantes como la frecuencia de la señal, el valor de cada señal en determinado momento del tiempo, la sincronización de las señales, la fase y muchas otras mediciones. Es por eso que el osciloscopio es un instrumento fundamental en los laboratorios.

Otras cosas que puede hacer: 

\begin{itemize}
    \item Análisis de fallas en circuitos electrónicos
    \item Decodificación de buses seriales y protocolos de comunicación
    \item Visualización de formas de onda de señales eléctricas
    \item Verificación del estado de instrumentos de medición
    \item Análisis de estados lógicos de circuitos digitales
\end{itemize}
\section{Generador de funciones}
Un generador de funciones es una fuente de señales que tiene la capacidad de producir diferentes formas de onda. La mayoría de los generadores de funciones pueden generar ondas senoidales, cuadradas y triangulares sobre
 un amplio rango de frecuencias (entre 0.01 Hz hasta cerca de 1 MHz).

El generador de funciones es un aparato electrónico que produce ondas sinusoidales, cuadradas y triangulares, además de crear señales TTL. Sus aplicaciones incluyen pruebas y calibración de sistemas de audio, ultrasónicos y servo. El generador de funciones, específicamente trabaja en un rango de frecuencias de entre 0.2 Hz a 2 MHz

\subsection{Onda alterna}
Se llama corriente alterna (CA) al tipo de corriente eléctrica más empleado domésticamente, caracterizado por oscilar de manera regular y cíclica en su magnitud y sentido. La manera más usual de representarla es mediante una gráfica (sobre un eje x/y) en forma de ondas sinusoidales.

\textbf{La corriente alterna es aquella que varía su magnitud en función del tiempo.} La forma de onda de corriente alterna más habitual es la senoidal. Con ella se consigue una transmisión más eficiente de la energía.

Su uso mayormente será para iluminación o TV pero debido a sus pérdidas por calor en el circuito interior de dicho inversor, para un consumo de 150W se necesitaría mínimo un inversor de 300W de potencia.

\subsection{Diente de sierra o triangular}
una onda de diente de sierra, clara y de sonido limpio, contiene armónicos pares e impares, así como el tono fundamental. Es ideal para crear sonidos de cuerda, colchón, bajo o viento-metal.
\subsection{Onda cuadrada}
Las ondas cuadradas son básicamente ondas que pasan de un estado a otro de tensión, a intervalos regulares, en un tiempo muy reducido. Son utilizadas usualmente para probar amplificadores (esto es debido a que este tipo de señales contienen en si mismas todas las frecuencias).

Buenas tardes ñañañañ


\end{document}