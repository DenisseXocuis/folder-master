\documentclass[letterpaper,12pt]{article}
\usepackage[utf8]{inputenc}

\makeatletter
\renewcommand{\@seccntformat}[1]{%
  \ifcsname specialformat#1\endcsname
    \csname specialformat#1\endcsname
  \else
    \csname the#1\endcsname\quad % default
  \fi
}
\makeatother

\newcommand{\specialformatsection}{}
\renewcommand{\thesubsection}{\arabic{subsection}}
\usepackage[T1]{fontenc}
\usepackage{charter}
\usepackage{geometry}
\usepackage{amsmath}
\usepackage{float}
\usepackage{graphicx}
\usepackage{subcaption}
\usepackage{amssymb}
\usepackage{adjustbox}
\usepackage{wrapfig} 
\usepackage{xcolor}
\usepackage{fancyhdr}
\usepackage{tabularx}

\title {\textbf{Francés :)}}
\author{Lara Xocuis Martha Denisse}
\date{6 de abril del 2024}
\geometry{top=2cm, bottom=2cm, left=2cm, right= 2cm} %%margen
\graphicspath{{images/}}
\parindent=0pt

\begin{document}
\maketitle
\thispagestyle{empty}
\newpage
\setcounter{page}{1}
\pagestyle{headings}

%%%%%%%%%%%%%%%%%%%%%%%%%%%%%%%%%%%%%%%%%%%%%%%%%%%%%%%%%%%%%%%%%%%%%%%%%%%%
\begin{sloppypar} 
\section{L'impératif}
Se utiliza para dar órdenes o consejos para una o más personas.

SOLO PARA: \textbf{TU, NOUS, VOUS} 

NO LLEVA SUJETO

\section{FUTUR PROCHE}
Para usar este tiempo verbal solo necesitas formarlo con el verbo aller (ir) en el presente, seguido de un infinitivo.


\section{Passé Composé}
\texttt{On l'utilise pour décrire des événements que se sont finis dans le passé.}
\begin{center}
    \begin{tabular}{|c|c|c|}\hline
        & Verbe Auxiliaire & \\
        Sujet + & être & + Participe Passé du verbe principal\\
        & avoir & \\
        & Conjugue au présent de l'indicatif & \\
        \hline
    \end{tabular}
\end{center}

\subsection{Maison D'être $\longrightarrow$ L'accord du genre et nombre}
\begin{itemize}
    \item naître
    \item mourir
    \item descendre
    \item arriver
    \item sortir
    \item venir
    \item entrer
    \item rentrer
    \item devenir
    \item aller 
    \item passer
    \item tomber
    \item monter
    \item retourner
    \item partir
    \item rester
    \item revenir
    \item \textbf{verbes pronominaux}
\end{itemize}
\textbf{Avec être:}\\ 
\texttt{Elle \underline{est partie} chez elle \\ Vous \underline{êtes partis} à l'école ce jour-là. \\ Elles sont parties en France}
\vspace{0.3cm}\\ 
\textbf{Le verbe "avoir" sont le reste des verbes:} \\ 
\texttt{J'ai eu une soeur \\ Elles ont mangé leurs fraises}

\subsection{Négation}
\subsubsection{Estructura}
\begin{center}
    \begin{tabular}{|c|c|c|c|c|} \hline
                &    & Auxiliaire &  & Verbe au \\
        Sujet& NE &  être &  PAS & Participe \\ 
                &    &  avoir &     & passé \\  
        \hline
    \end{tabular}
    \vspace{0.3cm}\\ 
    \texttt{Alain ne est pas parti hier}
\end{center}
\subsubsection{Verbes pronominaux}
\begin{center}
    \begin{tabular}{|c|c|c|c|c|c|} \hline
             && Pronom réflechi &&& \\
        Sujet&NE&Nous&Auxiliaire&PAS&Participe\\ 
             &&Vous&être&&Passé \\  
             &&Se&&& \\
\hline
\end{tabular}
\vspace{0.3cm}\\ 
\texttt{Je ne me suis pas réveillé tôt, hier}
\end{center}

\texttt{Nous \underline{ne sommes pas descendus} \textit{(descendre)} près de la Gare \\ Vous \underline{n'êtes pas devenus} \textit{(devenir)} médicins \\ Elles \underline{ne sont pas montées} \textit{(monter)} dans la voiture \\ Il \underline{n'est pas rentré} \textit{(rentrer)} chez le coiffeur}
\vspace{0.3cm}\\ 
\textbf{Pour le verbes pronominaux: }

\texttt{Lina et Mario \underline{ne se sont pas habillés} \textit{(habiller)} au même-temps \\ On \underline{ne s'est pas rasé} \textit{(raser)} dans le matin \\ Mari, Sven et moi, \underline{ne nous sommes pas preparés} \textit{(preparer)} pour l'examen}


\end{sloppypar}
\end{document}