\documentclass[letterpaper,12pt]{article}
\usepackage[utf8]{inputenc}

\usepackage[T1]{fontenc}
\usepackage{tgtermes} %%%font

\usepackage{geometry}
\usepackage{amsmath}
\usepackage{float}
\usepackage{graphicx}
\usepackage{subcaption}
\usepackage{amssymb}
\usepackage{adjustbox}
\usepackage{wrapfig} %%imagen envuelta por un texto
\usepackage{xcolor}
\usepackage{fancyhdr}
\usepackage{tabularx} %%TABLAS OH YEAH

\title {\textbf{Circuitos Lógicos}}
\author{Lara Xocuis Martha Denisse}
\date{5 de septiembre de 2023}
\geometry{top=2cm, bottom=2cm, left=2cm, right= 2cm} %%margen
\graphicspath{{images/}}
\parindent=0pt

\begin{document}
\begin{sloppypar}
\section{Sistemas numéricos}
\begin{itemize}
    \item Decimal: consta de 10 dígitos (0-9), ej, 19,24 
    \item Binario: consta de dos dígitos (0 y 1), ej, 1011, 1101
    \item Octal : consta de ocho dígitos (0-7), ej, 723, 126
    \item Hexadecimal: consta de 10 dígitos y seis letras, se denota de la sig manera:
    
    0 1 2 3 4 5 6 7 8 9

    \begin{table}[H]
        \centering
        \begin{tabular}{|c|c|c|c|c|c|}
            \hline 
            A & B & C & D & E & F \\ 
            \hline
            10 & 11 & 12 & 13 & 14 & 15 \\ 
            \hline
        \end{tabular}
    \end{table}

    Ej. A2F, 3F2C, x5B7, xEAF2C, 0x129, 0x5FEA2
\end{itemize}

BUenas tardes

\end{sloppypar}
\end{document}