\documentclass[letterpaper,12pt]{article}
\usepackage[utf8]{inputenc}
\usepackage{geometry}
\usepackage{amsmath}
\usepackage{float}
\usepackage{graphicx}
\usepackage{subcaption}
\usepackage{amssymb}
\usepackage{adjustbox}
\usepackage{wrapfig} %%imagen envuelta por un texto
\usepackage{xcolor}
\usepackage{fancyhdr}

\title {\textbf{Formularios inglés}}
\author{Deniso Xocuis}
\date{24 de julio 2023}
\geometry{top=2cm, bottom=2cm, left=2cm, right= 2cm} %%margen
\graphicspath{{images/}}
\parindent=0pt %%justificado

\begin{document}
\maketitle

%%%%%%%%%%%%%%%%%%%%%%%%%%%%%%%%%%%%%%%%%%%%%%%%%%%%%%%%%%%%%%%%%%%%%%%%%%%%
\begin{sloppypar} 
\section{Simple present}
Se utiliza para hablar acerca de hechos que ocurren en el momento en que se expresa la oración.

\section{Simple past}
Existen verbos regulares y verbos irregulares. 
\vspace{0.3cm}\\
\textbf{Excepciones: }
\vspace{0.3cm}\\
Para verbos que terminan en una "e", sólo añadimos -d. 

\textit{change} $\rightarrow$ \textbf{changed} 
\vspace{0.3cm}\\
Si el verbo termina en una vocal y una consonante (excepto "y" o "w"), doblamos la consonante final.

\textit{stop} $\rightarrow$ \textbf{stopped}
\vspace{0.3cm}\\
Con verbos que terminan en una consonante y una "y", se cambia por una "i".

\textit{study} $\rightarrow$ \textbf{studied}
\vspace{0.3cm}\\
Nota: Hay muchos verbos irregulares en inglés. Desafortunadamente, no hay una norma establecida para formarlos. A continuación tienes los tres verbos irregulares más comunes y los que actúan como verbos auxiliares.

be $\rightarrow$ \textbf{was/were}

do $\rightarrow$ \textbf{did}

have $\rightarrow$ \textbf{had}
\vspace{0.3cm}\\
\textbf{Pronunciación}
\vspace{0.3cm}\\
Pronunciamos la terminación "-ed" de forma diferente dependiendo de la letra que va al final del infinitivo. En general la "e" es muda. 
\vspace{0.3cm}\\
Con los infinitivos que terminan en "p", "k", "f" o "s" (excepto "t") pronunciamos la terminación "-ed" como una "t".

\textit{looked [lukt]}
\vspace{0.3cm}\\
Con los infinitivos que terminan en "b", "g", "l", "m", "n", "v", "z" (excepto "d") o una vocal, pronunciamos solo la "d".

\textit{yelled [jeld]}
\vspace{0.3cm}\\
Con los infinitivos que terminan en "d" o "t", pronunciamos la "e" como una "i"

\textit{ended [endid]}

\break \section{ESTRUCTURAS}

\begin{center}
    \textbf{PASADO SIMPLE}
\end{center}

\textbf{Affirmative Sentences }

Sujeto + was/were.... 
\vspace{0.3cm}\\
\textbf{Negative Sentences}

To be: Sujeto + "was/were" + not (wasn't/weren't)\dots






\end{sloppypar}
\end{document}