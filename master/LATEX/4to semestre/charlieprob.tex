\documentclass{article}
\usepackage[utf8]{inputenc}

% Cambiar el formato de secciones a números romanos
\renewcommand{\thesection}{\Roman{section}}

\title{Planteamiento del Problema: Basura Tecnológica en Veracruz}
\author{Nombre del Autor}
\date{\today}

\begin{document}

\maketitle

\section{Introducción}
Hoy en día en Veracruz, la mayoría de las personas tiene a la mano cualquier dispositivo electrónico, desde un teléfono o hasta una computadora. A consecuencia de este hecho, existe una gran desmedida de aparatos electrónicos que ha contribuido a un alto aumento de la "basura tecnológica" en el estado y en todo el mundo.

\section{El Problema}
En la región, cada cuatrimestre se tiene alrededor de 20 toneladas de basura tecnológica entre pantallas, bocinas y teléfonos celulares \cite{periodistasdigitales}.

Poco se habla de las consecuencias que pueden traer consigo la basura tecnológica al estar en contacto con las personas y el medio ambiente sin un manejo apropiado. No toda la población conoce que se usan ciertos metales y sustancias tóxicas para la fabricación de aparatos electrónicos con los que convivimos en nuestra vida cotidiana en los hogares; Veracruz también refleja la situación del país en cuanto a la basura electrónica; es decir, se desconoce la producción y el desecho de productos electrónicos en la entidad. Se cree que se generan altos niveles de contaminación, pero son difíciles de cuantificar, por lo que es necesario sumar esfuerzos para entender y abordar el problema actual y potencial de la generación y el manejo de la basura electrónica, que van en aumento. En nuestro país la cultura de reciclaje es muy pobre, por lo que es realmente importante impulsar a los ciudadanos a promover el reciclaje de la basura tecnológica y tratar de consumir dichos aparatos con responsabilidad.

\bibliographystyle{plain}
\bibliography{references}

\end{document}
