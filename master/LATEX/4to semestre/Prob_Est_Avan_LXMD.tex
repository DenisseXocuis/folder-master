\documentclass[letterpaper,12pt]{article}
\usepackage[utf8]{inputenc}

\makeatletter
\renewcommand{\@seccntformat}[1]{%
  \ifcsname specialformat#1\endcsname
    \csname specialformat#1\endcsname
  \else
    \csname the#1\endcsname\quad % default
  \fi
}
\makeatother

\newcommand{\specialformatsection}{}
\renewcommand{\thesubsection}{\arabic{subsection}}
\usepackage[T1]{fontenc}
\usepackage{charter}
\usepackage{geometry}
\usepackage{amsmath}
\usepackage{float}
\usepackage{graphicx}
\usepackage{subcaption}
\usepackage{amssymb}
\usepackage{adjustbox}
\usepackage{wrapfig} 
\usepackage{xcolor}
\usepackage{fancyhdr}
\usepackage{tabularx}

\title {\textbf{Actividad}}
\author{Lara Xocuis Martha Denisse}
\date{\today}
\geometry{top=2cm, bottom=2cm, left=2cm, right= 2cm} %%margen
\graphicspath{{images/}}
\parindent=0pt

\begin{document}
\maketitle
%%%%%%%%%%%%%%%%%%%%%%%%%%%%%%%%%%%%%%%%%%%%%%%%%%%%%%%%%%%%%%%%%%%%%%%%%%%%
\begin{sloppypar} 
\section{\LARGE EJEMPLO}
Imaginemos que un proveedor de servicios de Internet (ISP) afirma que sus planes premium tienen una velocidad media de conexión de al menos 100 Mbps (megabits por segundo). Un grupo de abonados realiza pruebas de velocidad a distintas horas del día durante una semana y registra los resultados.
\subsection*{Hipótesis nula y alternativa}
La hipótesis nula (H0) es que la velocidad media de conexión en las tarifas premium es igual o superior a 100 Mbps, y la hipótesis alternativa (H1) es que la velocidad media es inferior a 100 Mbps.
\vspace{0.3cm}\\ 
Por lo tanto, podemos describir:
\begin{center}
  \textbf{Hipótesis nula:} \\
$100 Mbps (\mu \geq 100)$
\vspace{0.3cm}\\ 
\textbf{Hipótesis alternativa:} \\
$100Mbps (\mu < 100)$
\end{center}
donde $\mu$ es la velocidad promedio del plan premium.

\subsection*{Conclusión}
En este contexto, el análisis de escenarios permite evaluar objetivamente en qué medida los servicios de internet cumplen las expectativas y promesas de los ISP, lo que pueden influir en las necesidades y preferencias de los usuarios en materia de contectividad en línea.

\end{sloppypar}
\end{document}