\documentclass[letterpaper,12pt]{article}
\usepackage[utf8]{inputenc}
\makeatletter
\renewcommand{\@seccntformat}[1]{%
  \ifcsname specialformat#1\endcsname
    \csname specialformat#1\endcsname
  \else
    \csname the#1\endcsname\quad % default
  \fi
}
\makeatother

\newcommand{\specialformatsection}{}
\renewcommand{\thesubsection}{\arabic{subsection}}

\usepackage[T1]{fontenc}
\usepackage{charter}

\usepackage{geometry}
\usepackage{amsmath}
\usepackage{float}
\usepackage{graphicx}
\usepackage{subcaption}
\usepackage{amssymb}
\usepackage{adjustbox}
\usepackage{wrapfig} %%imagen envuelta por un texto
\usepackage{xcolor}
\usepackage{fancyhdr}
\usepackage{tabularx} %%TABLAS OH YEAH

\title {\textbf{Redes de computadoras}}
\author{Lara Xocuis Martha Denisse}
\date{7 de Febrero del 2024}
\geometry{top=2cm, bottom=2cm, left=2cm, right= 2cm} %%margen
\graphicspath{{images/}}
\parindent=0pt

\begin{document}
\maketitle
\newpage
%%%%%%%%%%%%%%%%%%%%%%%%%%%%%%%%%%%%%%%%%%%%%%%%%%%%%%%%%%%%%%%%%%%%%%%%%%%%
\begin{sloppypar}

\section{Router de una computadoraAA}
Son computadoras que se especializan en el envío de \textbf{paquetes} a través de redes de datos. Son los responsables de la interconexión de redes. Seleccionan la mejor ruta para transmitir los pquetes y los reenvían al destino.
\vspace{0.3cm}\\ 
Router es capa 3 = red.
\vspace{0.3cm}\\
Los routers son el centro de una red, tienen 2 conexiones.
\begin{itemize}
    \item Conexión WAN 
    \item Conexión LAN
\end{itemize}

inserte dibujito de redes*

Tabla de enrutamiento

\subsection{Componentes de los routers y sus funciones} 
\begin{itemize}
    \item CPU 
    \item Memoria de acceso aleatorio (RAM)
    \item Memoria de sólo lectura (ROM)
    \item RAM no volátil (NVRAM)
    \item Memoria flash
    \item Interfaces
\end{itemize}

\subsection{Grupos principales de interfaces del router}
LAN:
\begin{itemize}
    \item Se usan para conectar el router a la red LAN 
    \item Tienen una dirección MAC de capa 2
    \item Se les puede asignar una dirección IP de capa 3
\end{itemize}
WAN:
\begin{itemize}
    \item oal chavo 
    \item tengo hambre
    \item mis fans le daran like y comentaran manseko haa
    \item tengo hambre we
    \item MORITE POBREEEEEEE
\end{itemize}

Cable $\rightarrow$ CSMA / CD
\vspace{0.3cm}\\  
WLAN $\rightarrow$ CSMA / CA

\section{Configuración de dispositivos y aplicación de direcciones}
Clock rate, para DCE, DTE

\section{Estructura de la tabla de enrutamiento}
Redes conectadas directamente: Correcponde a un dispositivo conectado a otra interfaz del router
\vspace{0.3cm}\\ 
Redes conectadas de forma remota: Una red que no está conectada directamente 

\section{Rutas estáticas de la tabla de enrutamiento (rutamiento estático)}
\begin{itemize}
  \item Incluyen: Dirección de \textbf{red}, \textbf{máscara de subred} y dirección IP del router del siguiente salto o interfaz de salida.
  \item Se ven en la tabla de enrutamiento con el código S. 
  \item Antes de configurar se deben tener redes conectadas
\end{itemize}

¿Cuándo implementar enrutamiento estático? pocos routers, cuando es una red pequeña, cuando está conectada a internet sólo a través de un ISP
\vspace{0.3cm}\\ 
dinámico : para agregar redes remotas a una tabla de enrutamiento, para detección de redes.
\vspace{0.3cm}\\ 
Protocolos de enrutamiento: RIP, OSPF 

\section{Campos de paquetes de IP}
\section{Rutas de routers y conmutación de paquetes}
\section{Rutas de routers y conmutación de paquetes}
\section{Comandos c:}
ARP, clock rate, show ip config
\newpage
\section{Formulariosssss yeah}

\end{sloppypar}
\end{document}